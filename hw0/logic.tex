\documentclass{article}

\usepackage{geometry}
\usepackage{url}

\title{Alexander Friedman's assumption}  % Replace with your title
\author{Nirmal Advani} % Replace with your name

\begin{document}
\maketitle

\section{The Text}

\begin{quotation}
  ``Will the universe eventually stop expanding and start contacting, or will it expand forever? To answer this question we need to know the present rate of expansion of the universe an its present average density. If the density is less than a certain critical value, determined by the rate of expansion, the gravitation attraction will be too weak to halt the expansion. If the density is greater than the critical value, gravity will stop the expansion at some time in the future and cause the universe to recollapse.
  if we a up the masses of all stars that we can see in our galaxy an other galaxies, the total is less than one hundredth of the amount required to halt the expansion of the universe, even for the lowest estimate of the rate of expansion. Our galaxy and other galaxies, however, must contain a large amount of "dark matter
  " that we cannot see directly. When we add up all this dark matter, we still get only about one tenth of the amount required to halt the expansion."
  
\end{quotation}

The text is taken from the Book, ``The Brief History of Time'', by Stephen Hawking that appeared on pages 48-49.
\section{Inferred Information}

\begin{itemize}
\item Universe is expanding.
\item Gravity can affect the expansion of universe.
\item Gravity can recollapse the universe.
\item There is dark matter in galaxies.
\item Darkmatter ad up to the mass.
\item The density of our universe is less than certain critical value.
\end{itemize}

\section{The Argument}

There is one conclusion for both argument
 
Premise is that the density of our universe is less than certain critical value.

conclusion :
\begin{itemize}
\item The mass of all galaxies and dark matter is not enough to halt the expansion of universe.
\end{itemize}


According to the arguments, the expansion of universe will not halt if its density is less than certain critical value and if the density is greater than critical value, the expansion will halt and universe will recollapse. Since the density is less than critical value the gravity can not halt the expansion which makes conclusion of first argument true and conclusion of second argument false. But both arguments are valid since they come to different conclusions provided the premise is true. 

\end{document}
